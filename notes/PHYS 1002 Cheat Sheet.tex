\documentclass{article}
\usepackage{graphicx} % Required for inserting images
\usepackage{amsmath}
\usepackage{amssymb}
\usepackage{gensymb}
\usepackage{titlesec}
\usepackage[a4paper,total={8.1in,11.25in}]{geometry}
\titleformat{\section}{\bfseries}{Chapter\,\arabic{section}}{1em}{}
\titleformat{\subsection}{\bfseries}{(\alph{subsection})}{1em}{}
\setlength{\parindent}{0.0in}
\titlespacing{\section}{0pt}{0pt plus 3pt minus 2pt}{0pt plus 3pt minus 2pt}
\DeclareMathOperator{\arcsec}{arcsec}
\DeclareMathOperator{\rad}{rad}
\DeclareMathOperator{\s}{s}
\DeclareMathOperator{\h}{h}
\DeclareMathOperator{\days}{days}
\DeclareMathOperator{\yrs}{yrs}
\DeclareMathOperator{\nm}{nm}
\DeclareMathOperator{\mum}{\mu m}
\DeclareMathOperator{\mm}{mm}
\DeclareMathOperator{\cm}{cm}
\DeclareMathOperator{\m}{m}
\DeclareMathOperator{\km}{km}
\DeclareMathOperator{\AU}{AU}
\DeclareMathOperator{\ly}{ly}
\DeclareMathOperator{\pc}{pc}
\DeclareMathOperator{\Mpc}{Mpc}
\DeclareMathOperator{\g}{g}
\DeclareMathOperator{\kg}{kg}
\DeclareMathOperator{\atm}{atm}
\DeclareMathOperator{\K}{K}
\DeclareMathOperator{\J}{J}
\DeclareMathOperator{\W}{W}


\title{PHYS 1002 Notes}
\author{Kwok Yuk Hang}
\date{November 2023}

\begin{document}
$\mathcal{L}$: Luminosity \quad $\mathcal{T}$: Temperature \quad $\mathcal{D}$: Density \quad $\mathcal{P}$: Pressure \quad $\mathcal{M}$: Mass
\section{A Modern View of the Universe}
\textit{Cosmic Address}: \textit{Universe: Filaments and Voids: Local Supercluster: Local group: Milky Way Galaxy: Solar system: Earth: Asia:}\\
\text{Planet}: Rocky, icy, or gaseous object that orbits a star, shining by reflected light. $d_{\text{Earth}}=0.009d_{\text{Sun}}\qquad d_{\text{Jupitar}}=0.1d_{\text{Sun}}$\\
\textit{Satellite}: Man-made object that orbits a planet.\\
\textit{Asteroid}: Relatively small and \textit{rocky} object that orbits a star, often non-spherical in shape.\\
\textit{Comet}: Relatively small and \textit{icy} object that orbits a star. Comets usually have two tails: plasma tail and dust tail.\\
\textit{Solar system}: Stars and all the material that orbits it, including its planets and moons.\\
$1:10^{10}$ scale: $d_{\text{Sun}}'=0.14\m$ (Grapefruit), $d_{\text{Earth}}'=0.0012\m$ (Tip of ballpoint pen) ($15\m$ away), $d_{\text{Neptune}}'=0.0048\m$ ($449\m$ away).\\
\textit{Nebula}: Interstellar cloud of gas and/or dust\\
$1:10^{19}$ scale: distance to nearest star $=0.004\m$.\\
Currently, $d_{\text{observable universe}}\approx 9.3\times 10^{10}\ly$.\\
No. of galaxies $\approx 10^{11}$. No. of stars $\approx 10^{11}\times 10^{11}=10^{22}$. As many stars as grains of sand on Earth's beaches.\\
Universe is at least 250 times larger than observable universe.\\
Look-back time (LBT): Time elapsed between when light was emitted and when we detect the light on Earth.\\
Speed of light $c=3\times 10^{5}\km\s^{-1}$\qquad Light year: $1\ly\approx 9.46\times10^{12}\km\approx 10^{13}\km$\\
The farther away we look in distance, the further back we look in time.\\
Cosmic Calendar: A scale on which we compress the history of the Universe into 1 year.\\
Human civilization: Few seconds. Human lifetime: Very small fraction of a second.\\
Earth rotates around its axis once every day. Earth orbits the Sun once every year. Average distance $=1\AU\approx 1.5\times 10^{8}\km$.\\
Earth's axis tilted by $23.5\degree$ pointing to Polaris, rotates in same direction it orbits (counterclockwise viewed from above North Pole)
Earth spins at $1650\km\h^{-1}$ about its axis and orbits at $\sim 107,000\km\h^{-1}$ around the Sun.\\
We don't feel Earth's motions due to almost constant and smooth motions and low angular speeds ($0.04$ turns $\h^{-1}$) $F=\frac{mv^{2}}{R}$.\\
Sun moves randomly at $>70,000\km\h^{-1}$ relative to other stars in local solar neighbourhood.\\
Sun orbits the center of galaxy every $230\times 10^{6}\yrs$ with speed $\approx 800,000\km\h^{-1}$.\\
All stars appear to be stationary and fixed due to large distance between stars.\\
All galaxies outside our Local Group are moving away from us. The more distant the galaxy, the faster it is racing away.
\section{Discovering the Universe for Yourself}
\textit{Constellations}: Regions of the sky. Total $88$. Star appears to go around once per day.\\
\textit{Celestial Sphere (c.s.)}: A model which helps us visualize positions of objects in the sky. Imaginary sphere with Earth as center.\\
Sphere appears to rotate about celestial pole axis once a day due to Earth's rotation.\\
Sun, moon and planets appear to move from one place to another on the sphere.\\
\textit{Zenith}: Point on c.s. directly overhead \qquad \textit{Nadir}: Point on c.s. directly underneath \qquad \textit{Horizon}: Plane with zenith as normal\\
Celestial equator: projection of Earth’s equator onto the c.s.\\
\textit{North (South) celestial pole}: projection of Earth’s north (south) pole onto the c.s.\\
Milky Way in c.s.: A band of light that makes a circle around c.s..\\
Milky Way is flat spiral galaxy. Solar system is half way between center and edge.\\
\textit{Meridian}: Half circle from N horizon through zenith to S horizon\\
Axial tilting effects: Longer day in summer and shorter in winter, Sun's altitude changes with seasons\\
Cause of Analemma (figure-8 shape): Earth's $23.5\degree$ axial tilt and eliptical orbit\\
More spread light: Lower $\mathcal{T}$ (Distance doesn't matter as variation of distance is small ($3\%$))\\
At equinoxes: both hemisphere's receive same amount of sunlight\\
At summer/winter solstice: axis tilt at largest angle with northern/southern hemisphere towards the sun (Tropic of Cancer/Capricorn)\\
$4$ seasons only at temperate latitudes (TLs) (Between tropics and polar regions $23.5-66.5\degree$)\\
Tropics: Region surrounding the Equator\\
Gravity of moon, sun, other planets causes Earth's axis to precess (over about $26,000\yrs$) Polaris won't always be North ($22.1-24.5\degree$)\\
Phases of Moon (Light from empty, appear start in right): Rise: $6$ a.m., Highest: noon, Set: $6$ p.m., adding $3\h$\\
New Moon, Waxing Crescent, First Quarter, Waxing Gibbous, Full Moon, Waning Gibbous, Third Quarter, Waning Crescent\\
Sidereal month $=27.3\days$ \qquad Synodic month (Full cycle) $=29.5\days$\\
\textit{Umbra}: Region of total darkness \quad \textit{Penumbra}: Region of partial darkness\\
Lunar eclipses: full moon\\
Total eclipses: Red moon (violet scattered away, red is refracted), completely immersed in umbra \quad Partial \quad Penumbral\\
Solar eclipses (new moon): Total (totality reached, sees corona), annular (moon smaller than sun), partial\\
Moon's orbit tilted $5\degree$ with Earth's orbit. \quad Line of nodes: Moon on Earth's orbit plane (Possible eclipse when point towards Sun)\\
Apparent retrograde motion: Planets move westward instead of eastward
\section{Ancient Science of Astronomy}
Modern science trace roots to Greeks: Developed models of nature and emphasized predictions of models agree with observations\\
Ptolemaic model had each planet move on small circle whose cener moves around Earth on larger circle\\
Kepler's First Law: Orbit of each plenet around Sun is ellipse with Sun at one focus\\
\textit{Perihelion}: Point nearest to Sun \quad \textit{Aphelion}: Point farthest to Sun\\
Kepler's Second Law: Line joining planet and sun sweeps out equal area in equal time (Planets travel faster when closer to sun)\\
Kepler's Third Law: $p^{2}=a^{3}$ ($p$: Orbital period in yrs, $a$: Semi-major axis in $\AU$)\\
Copernicus: Sun-centered model. Tycho provides data to improve the model. Kepler found a model that fit Tycho's data.\\
Galileo: Reject: Nature of motion, heavenly perfection, parallax
\newpage
\section{Understanding Motion, Energy and Gravity}
Gravitational constant $G=6.67\times 10^{-11}\m^{3}\kg^{-1}\s^{-2}$ \quad $a=\frac{v^{2}}{r}=\frac{G\mathcal{M}}{r^{2}}$\\
Newton's version of Kepler's Third Law: $p^{2}=\frac{4\pi^{2}}{G(\mathcal{M}_{1}+\mathcal{M}_{2})}r^{3}$ \quad Assume $\mathcal{M}_{1}=0$ when $\mathcal{M}_{1}<<\mathcal{M}_{2}$\\
Period of orbit: $p=\frac{2\pi r}{v}$ \quad Angular momentum: $L=\mathcal{M}r\times v$ \quad Torque: $\tau=rF\sin{\theta}$\\
Gravitational Potential Energy: $\text{PE}=\mathcal{M}gh=-\frac{G\mathcal{M}_{1}\mathcal{M}_{2}}{r}$ \quad Mass Potential Energy: $E=\mathcal{M}c^{2}$\\
Change orbit: Object lose or gain orbital energy \quad Change by: Atmospheric drag, gain orbital energy, gravitational encounter\\
Escape velocity: $v_{\text{escape}}=\sqrt{\frac{2G\mathcal{M}}{r}}$\\
Tides: Periodic variation in height of Earth's ocean (Mainly due to tidal force of moon, sun slightly)\\
Tidal force (differential force): Non-uniform gravity \quad Vertical: Stretching force \quad Horizontal: Compressing force\\
Tidal force on Earth is hardly noticeable. That on stellar black hole is very large (Spaghettification)\\
High tide near and far side (Moon pulls Earth away from water on far side) \quad Sun do only $\frac{1}{3}$\\
Tidal Friction: Earth's self-rotation drag tidal bulges around with it (Slowing Earth's rotation)\\
Gravity of bulges pulls Moon ahead: increasing orbital distance ($0.0378\m$ per yr)\\
Tidal locking: Tidal friction slow Earth's down trying to have rotation be equal to orbital motion of Moon\\
Increase Earth-day from $5-6\h$ to $24\h$, kocking of Moon is faster due to less massive
\section{Light, The Cosmic Messenger}
Light is electromagnetic wave \quad Light speed: $c=3\times 10^{8}\m\s^{-1}=f\lambda$ \quad Low $f$: Radio wave \quad High $f$: Gamma rays\\
Photons: particles of EM radiation \quad Photon energy $E=hf$ ($h=6.626\times 10^{-34}\J\s$) Photon momentum $p=\frac{E}{c}=\frac{h}{\lambda}$\\
Atom: $d_{\text{H nucleus}}=1.75\times 10^{-15}\m$, $d_{\text{U nucleus}}=15\times 10^{-15}\m$\\
Atomic no.: No. of protons in nucleus \quad Atomic Mass No.: No. of protons and neutrons\\
Isotopes: Sane no. og protons but different no. of neutrons \quad Molecules: Consist of $\geq 2$ atoms\\
Matter emit. absorb, transmit, reflect and scatter light\\
Three basic types of spectra:\\
Continuous spectrum: Spectrum span all visible $\lambda$\\
Emission Line Spectrum: Excited thin or low $\mathcal{D}$ cloud emit lights at specific $\lambda$ (Bright emission lines)\\
Absorption Line Spectrum: Cold, thim or low $\mathcal{D}$ cloud absorb lights at specific $\lambda$ (Dark absorption lines)\\
Specific atom has unique set of energy level: Downward transitions: Produce emission lines \quad Upward: Produce absorption lines\\
Use spectral fingerprints to determine chemical compositions\\
Thermal radiation: all thing above absolute zero ($-273\celsius$) emit thermal radiation\\
Stefan-Boltzmann's Law: Flux $F=\sigma\mathcal{T}^{4}$ ($\sigma=5.7\times 10^{-8}\W\m^{-2}\K^{-4}$) \qquad Wien's Law: $\lambda_{\text{max}}=\frac{2.9\times 10^{6}}{\mathcal{T}}\nm$\\
Luminosity: Total amount of light energy given out by star per second\\
Doppler Effect: $\lambda$ of wave received by observer is changed when source is moving with respect to observer\\
$\frac{\Delta\lambda}{\lambda_{\text{rest}}}=\frac{v_{r}}{c}$ $v_{r}$: radial velocity of source relative to observer\\
Blueshift: Object moving towards us \qquad Redshift: Object moving away from us
\section{Telescopes: Portals of Discovery}
Telescope collect more light than eyes: Larger light-collecting area\\
Telescope see more detail than eyes: Higher angular resolution\\
Some telescopes detect light that is invisible in eyes: Broader spectral coverage\\
Bigger telescope: Higher light gathering power (no. of photons collected, proportional to square of diameter), higher resolving power\\
Angular resolution: Minimum angular separation to distinguish\\
Resolving power $\alpha=\frac{1.22\lambda}{D}$ \quad $D$: Aperture (Diameter of lens)\\
Refracting telescope: Chromatic aberration (Lens will not focus different color in same place since index of refraction varies with $\lambda$)\\
Reflecting telescope: E.g. Keck I/II\\
Radio telescope: Big dish due to weak radio signal from space and long $\lambda$ (Low resolving power)\\
Telescope in space: No light pollution, no air movement blurs (Turbulence cause twinkling), no atmosphere absorb EM spectrum (X-ray, gamma ray, far-UV)\\
Adaptive optics: Rapid change in mirror shape to compensate for atmospheric turbulence (Use bright star as reference)\\
Interferometry: Combine signals detected from no. of radio telescopes. $D$ becomes distance between telescopes
\section{Our Star}
$r_{\text{Sun}}=6.9\times 10^{8}\m$ \quad $\mathcal{M}_{\odot}=\mathcal{M}_{\text{Sun}}=2\times 10^{30}\kg$ \quad $\mathcal{L}_{\odot}=3.8\times 10^{26}\W$ \quad Source of energy: Nuclear energy (Nuclear fusion) $E=\mathcal{M}c^{2}$\\
Nuclear fusion: Need $>10\times 10^{6}\K$, high $\mathcal{D}$, high $\mathcal{P}>10^{10}\atm$ to overcome electric repulsion\\
Gravitational contraction: Provide energy for fusion \\
Stellar Balance: Hydrostatic equilibrium (Gravity inward and internal pressure outward)\\
Pressure-temperature thermostat: Rely on strong $\mathcal{T}$ dependence of fusion rate\\
Core contracts and gets hotter, fuse H faster, more radiation and pressure, core expands, core gets cooler, reaction decrease\\
Structure of Sun: Solar wind: Flow of charged particles from surface \qquad Corona: Outermost layer of solar atmosphere ($1\times 10^{6}\K$)\\
Chromosphere: Middle layer of solar atmosphere ($10^{4}-10^{5}\K$) \qquad Photosphere: Visible surface of Sun ($5800\K$)\\
Convection zone: Energy transport upward by rising hot gas (Bottom $1.5\times 10^{6}\K$)\\
Radiation zone: Energy transported upward by photons $7-1.5\times 10^{6}\K$ \qquad Core: Energy generated by nuclear fusion ($15\times 10^{6}\K$)\\
Heat transfer: Radiation, convection \quad depends on $\mathcal{T}$, thermal and density gradients, and $\mathcal{P}$\\
Scattering: Radiation zone: less-absorptive \quad Convective zone: more absorptive (Takes photon $1\times 10^{6}\yrs$ to go out)\\
$<0.4\mathcal{M}_{\odot}$: Star conductive \quad $\approx 1\mathcal{M}_{\odot}$: Core, radiative, convective \quad $>>1\mathcal{M}_{\odot}$: Convective core, radiative shell\\
Solar neutrino: electron neutrino ($v_{e}$), muon neutrino  ($v_{\mu}$), tau neutrino ($v_{\tau}$)\\
Sunspots ($4500\K$): Cooler part of surface (Strong magnetic fields) \quad Zeeman effect: Split spectral line into 3 lines\\
Strong magnetic activity causes solar prominences, magnetic storm causes solar flares\\
Coronal mass ejection (huge, balloon-shaped plasma bursts): Send bursts of energetic charged particles out through solar system
\newpage
\section{Surveying the Stars}
Luminosity: Total amount of power a star radiates \qquad Apparent brightness: Amount of starlight that reaches Earth\\
Brightness $B=\frac{\mathcal{L}}{4\pi d^{2}}$ (Inverse Square Law)\\
Parallax: Apparent shift in location of object with respect to ones further away due to change in observation point\\
Stellar parallax $\theta$: Half of total shift in angle of that star using Earth's orbital diameter (Works for nearby stars)\\
$1\pc=206265\AU=3.26\ly$ \qquad $1\rad=1\arcsec$ \qquad $d=\frac{1}{\theta}$ \quad $\theta:$ in $\arcsec$ \quad $d:$ in $\pc$\\
Stars luminosity: $10^{-4}-10^{6}L_{\text{Sun}}$ \quad Dim stars are far more common\\
Apparent magnitude $m$: Magnitude $1$ stars (first class): brightest \quad Magnitude $6$ stars (sixth class): faintest\\
Ratio: $\frac{B_{A}}{B_{B}}=(100^{\frac{1}{5}})^{m_{B}-m_{A}}$\\
Absolute magnitude $M$: $\mathcal{L}$ of stars \qquad $\frac{\mathcal{L}_{A}}{\mathcal{L}_{B}}=2.512^{M_{B}-M_{A}}$ \qquad Distance modulus $=m-M$ \quad Distance in $\pc$ $d=10^{0.2(m-M+5)}$\\
Measure surface $\mathcal{T}$: $\mathcal{L}$, level of ionization (absorption line) \quad Surface: ($50000\K$) O B A F G K M ($3000\K$)\\
If $\mathcal{T}$ is too high, most or all H atoms have electrons staring out at level 2.\\
If $\mathcal{T}$ is too low, most of H atoms have electrons on ground state.\\
Optimized $\mathcal{T}$ ($10000\K$ A star) at which absorption line strength hits maximum\\
Stellar mass: Need $2$ of $3$ observables: orbital period ($p$), orbital separation ($a$ or $r$), orbital velocity ($v$) $v=\frac{2\pi r}{p}$\\
Types of Binary star systems: Visual Binary: Pair of stars that we see distinctly\\
Eclipsing Binary: Cchange in apparent rightness \qquad Spectroscopic Binary: Measuring blueshift (Approach) and redshift (Recede)\\
Eclipsing binary offers orbital period. Spectroscopic binary offer orbital velocity and period, thus the $\mathcal{M}$ of heavier star.\\
Most massive star: $226\mathcal{M}_{\odot}$ \qquad Least massive star: $0.08\mathcal{M}_{\odot}$\\
Hertzsprung-Russell (HR) diagram: Plot $\mathcal{L}$ and surface $\mathcal{T}$ of stars on a graph\\
Radii of stars: Cannot be measured directly due to too small angular diameter, use $\mathcal{L}=4\pi\sigma\mathcal{T}^{4}R^{2}$ \quad Size of stars: $0.01r_{\text{sun}}-2000r_{\text{sun}}$\\
Stars with low $\mathcal{T}$ and high $\mathcal{L}$: Giants and supergiants\\
Stars with high $\mathcal{T}$ and low $\mathcal{L}$: White dwarfs\\
Collisional broadening: Broadening of spectral lines increases with frequency of collisions which increases $\mathcal{D}$ of atmosphere of stars\\
Uncertainty principle: $\Delta x\Delta p\geq\frac{\hbar}{2}$\\
Supergiants: Larger in size $\rightarrow$ Lower $\mathcal{D}$ in atmosphere $\rightarrow$ Sharper spectral lines\\
Same surface $\mathcal{T}$: Sharp to broader lines, high to low $\mathcal{L}$: supergiants, giants, main sequence, white dwarf\\
Star's full classification includes spectral type and $\mathcal{L}$ class E.g. Sun (G2 V)\\
Main-Sequence (MS) Line: Luminous stars are hot (blue), less luminous ones are cooler (red) \quad Most stars fall on lower ends\\
Core $\mathcal{P}$ and $\mathcal{T}$ of more massive MS star is higher to balance gravity $\rightarrow$ Greater $\mathcal{L}$, higher surface $\mathcal{T}$, larger radius\\
$\mathcal{M}$ and $\mathcal{L}$ of MS stars: $\mathcal{L}=\mathcal{M}^{3.5}$ \qquad $L$: In terms of $\mathcal{L}_{\odot}$ \quad $\mathcal{M}$: In terms of $\mathcal{M}_{\odot}$\\
Life expectancies of stars: $t=\frac{\mathcal{M}}{\mathcal{L}}=\mathcal{M}^{-2.5}$ \quad Cool stars live longer\\
MS Star after fusion has ceased: $>0.25\mathcal{M}_{\odot}$: Giant and supergiant \quad $<0.25\mathcal{M}_{\odot}$: Blue dwarf $\rightarrow$ white dwarf $\rightarrow$ black dwarf\\
Star clusters: Group of stars born around the sane time at about same distance to Earth (Determine age and distance)\\
Two types of star clusters: Globular clusters (round, very dense and old) and open clusters (Random, not dense, younger)\\
Massive stars move out of MS sooner than lower $\mathcal{M}$ star \qquad More massive stars begin leaving MS in cluster over time\\
Turn-off point: All MS stars above the point have moved away \qquad Age of cluster $t=\mathcal{M}_{\text{turn-off}}^{-2.5}$\\
Distance: Using standard HR diagram for absolute magnitude ($M$) and using HR diagram for apparent magnitude ($m$)
\section{Life of Stars}
Interstellar Medium: Nebulae (visible/invisible parts of molecular clouds) between stars\\
See nebulae in 3 ways:\\
Emission nebulae: Stars are embedded in/near nebula, emitting ultraviolet that excite H to give emission spectra (Pink-red in general)\\
Reflection nebulae: Scattering of blue starlight by dust (Blue)\\
Dark nebulae: Dense clouds of gas and dust blocking light of distant stars\\
Interstellar reddening: Stars visible near the edges of dark nebula are reddened (Red slightly reduced, blue greatly reduced)\\
Formation of stars: Start at molecular cloud ($\mathcal{T}\approx 30\K$, $n\approx 300\cm^{-3}$) contain at least few hundred $\mathcal{M}_{\odot}$ for gravity to overcome $\mathcal{P}$\\
Prevent $\mathcal{P}$ buildup via emitting infrared, microwave, radio photons generated from gravitational contraction $\rightarrow$ further contraction\\
Stars begin to form: Dust grains that absorb visible light emit infrared\\
Conservation of energy: Cloud heats up when it contracts. Continue contraction if part of thermal energy is radiated away\\
Conservation of angular momentum: Cloud becomes smaller cause it to spin faster\\
Flattening: Gas settles into spinning disk as spin hampers collapse perpendicular to spin axis and collision between particles\\
Formation of jets: Rotation causes bipolar jets of matter to shoot out (Due to dynamic interactions in accretion disc)\\
Accretion discs (Thought) generate tangled/twisted magnetic fields that collimate jets\\
Protostar: Fragment collapses under self-gravity. Very high $\mathcal{L}$ due to gravity, $\mathcal{T}$ too low for nuclear reaction\\
MS Star: Protostar contracts and heats until $\mathcal{T}$ is enough for H fusion \quad Contraction ends at hydrostatic equilibrium\\
$\mathcal{L}$ drops and shell shrinks, surface $\mathcal{T}$ increases and size drops\\
Upper limit: $150\mathcal{M}_{\odot}$: So luminous that collective pressure of photons drives matter into space\\
Lower limit: $<0.08\mathcal{M}_{\odot}$ (Brown dwarf): Fusion cannot begin as $T<10^{7}\K$\\
Degenerate electron gas: Pauli exclusion principle: No 2 electrons occupy same state \qquad Degeneracy pressure: Do not depend on $\mathcal{T}$\\
Brown dwarf: Electron degeneracy pressure halts contraction $\rightarrow$ no fusion (Emit infrared due to heat left from contraction)\\
Loss $\mathcal{L}$ over time to become black dwarf \quad Infrared observation can reveal recently formed brown dwarfs\\
Born star: begins at zero-age main sequence (ZAMS) line in HR diagram \quad Older star: More $\mathcal{L}$ and cooler surface $\mathcal{T}$\\
Life of star: $4$H $\rightarrow 1$He \quad Use up H and core contracts $\rightarrow$ more luminous and surface expands $\rightarrow$ cooler surface $\mathcal{T}$\\
Medium star: H shell heats up and do H fusion (hydrogen shell burning) $\rightarrow$ more He, He core contracts, becomes red giant\\
Red giant: Higher $\mathcal{L}$ and lower surface $\mathcal{T}$ \quad H burning shell deposits He ash into He core\\
He core increases in $\mathcal{M}$ and contracts until electrons become degenerate for red giants with $0.25\mathcal{M}_{\odot}<\mathcal{M}<2.25\mathcal{M}_{\odot}$\\
Core $\mathcal{T}$ goes up with no change in $\mathcal{P}$, degenerate $\mathcal{P}$ increase with $\mathcal{D}$ as more He ash drops into core\\
For medium-mass star, core $\mathcal{T}$ is high enough for He fusion \quad Triple-$\alpha$ process: $3$ $^{4}$He $\rightarrow 1$ $^{12}$C\\
At higher core $\mathcal{T}$ ($2\times 10^{8}\K$), $^{12}$C$+$ $^{4}$He $\rightarrow$ $^{16}$O\\
Helium flash: He fusion starts in degenerate core with broken thermostat, fusion rate skyrocket until enough $\mathcal{T}$ to eliminate degeneracy\\
Core expands, electron gas $\mathcal{D}$ drops and stop degenerate\\
Helium core fusion: Burn He in core and H in shell $\rightarrow$ Core expands, absorb energy from envelope $\rightarrow$ Contracts, hotter and smaller\\
C and O produced from He fusion is left behind around the center\\
Double shell-burning: He runs out at core, C and O core contracts and heats up $\rightarrow$ He shell fusion + H shell fusion $\rightarrow$ Red giant\\
Hydrostatic equilibrium is impossible, outer layers expelled by high radiation and thermal pressure $\rightarrow$ Planetary nebula\\
Core left behind is white dwarf\\
CNO cycle: High mass MS star fuse H to He using C, N, O as catalysts \quad Greater core $\mathcal{T}$ enables H to overcome greater repulsion\\
Life of high-mass star: High core $\mathcal{T}$ allow He to fuse with heavier elements \quad Core $\mathcal{T}$ in star $>\mathcal{M}_{\odot}$ allow fusion of elements to Fe.\\
Fe is dead end because Fe has lowest mass per nuclear particle \quad Builds up in core until degeneracy pressure cannot resist gravity\\
Supernova explosion: Core collapses due to electrons combine with protons, making neutrons and neutrino\\
$1.4-3\mathcal{M}_{\odot}$: Neutron star \qquad $>3\mathcal{M}_{\odot}$: Black hole
\section{Bizarre Stellar Graveyard}
White dwarf: Remaining cores of dead medium and low mass star \quad Electron degeneracy pressure supports against gravity\\
Ideal gas law: $\mathcal{P}V=nR\mathcal{T}$ when particles have much room to move\\
For degenerate matter, gravitational contraction can be balanced by degeneracy pressure\\
White dwarfs mainly made of C and O from medium mass star, mainly made of He from low mass star \quad Size similar to Earth\\
Chandrasekhar Limit: $M<1.4M_{\odot}$ takes into account of relativity \quad $>1.4M_{\odot}$ allows gravitational collapse\\
Size of white dwarf decreases with its $\mathcal{M}$ \quad Size goes to $0$ when $\mathcal{M}=1.4\mathcal{M}_{\odot}$\\
By $\Delta p\Delta x\geq\frac{\hbar}{2}$, as white dwarf's $\mathcal{M}\to 1.4\mathcal{M}_{\odot}$, electrons move nearly speed of light\\
Stars near white dwarf: Matter orbits white dwarf in accretion disk due to angular momentum \\
Friction between rings of matter heat up the disk and glow \quad surface of white dwarf hot enough for H fusion \\
Fusion begins suddenly and explosively, causing nova \quad Nova system temporarily appear much brighter, drives accreted matter out\\
White-dwarf supernova (Type Ia): C and O fusion begins as white dwarf reaches Chandrasekhar limit, causing complete explosion\\
Release energy larger than gravitational binding energy $\rightarrow$ nothing in center\\
Massive-star supernova (Type II): Iron core reaches white dwarf limit and becomes neutron star or black hole\\
In light curves, type II has plateau in luminosity while type Ia doesn't \quad Type Ia is much more $\mathcal{L}$\\
Type Ia do not have H absorption lines due to H being consumed by nova and become He\\
Type Ia show strong ionized silicon emission line at $615\nm$\\
Neutron star: Ball of neutrons from massive-star supernova \quad Supported by degeneracy pressure\\
Pulsar: Neutron star that beams radiation along magnetic axis that is not aligned with rotation axis \quad Emits in form of 2 beams\\
Source: Non-thermal synchrotron radiation: Emitted from particles accelerated rapidly along magnetic field\\
Thermal radiation: Particles colliding with neutron star surface at magnetic poles \\ 
Contain x-rays, optical and radio radiation since protons smash at extremely high velocities\\
Why Pulsars are neutron stars: Spin rate of fast pulsars $=1000$ cycles per second \quad Surface rotation velocity $=60000\km\s^{-1}$\\
Matter accreting onto neutron star becomes hot and dense enough for H and He to fuse $\rightarrow$ Produce burst of x-rays\\
Neutron star has thin atmosphere of He, accreting matter contain He and H fusion forms He\\
Neutron degeneracy pressure has a limit of $3\mathcal{M}_{\odot}$ $\rightarrow$ becomes black hole\\
Escape velocity: $v_{\text{escape}}=\sqrt{\frac{2G\mathcal{M}}{r}}$\\
Schwarzschild Radius: Smallest distance from black hole such that light can escape \quad $r_{s}=\frac{2G\mathcal{M}}{c^{2}}$\\
Non-rotating black hole: Singularity at center (0 size and infinite $\mathcal{D}$) \quad Event horizon: singularity centered with Schwarzschild radius\\
Tidal force for force at top and force at bottom: $F_{1}-F_{2}=\frac{2G\mathcal{M}_{\text{black hole}}\mathcal{M}\ell}{r^{3}}$\\
At event horizon, $F_{1}-F_{2}=\frac{c^{6}\mathcal{M}\ell}{4G^{2}\mathcal{M}_{\text{black hole}}}$ \quad More massive black holes have smaller tidal force\\
Black hole strongly warps space-time in vicinity of event horizon\\
Gravitational Time Dilation: Clock near massive object appears to run more slowly\\
Gravitational redshift: Light is redshifted when moving away from massive object\\
Methods for hunting black holes:\\
Measure mass: Use orbital properties of companion or measure velocity and distance or orbiting gas \\ 
Compact object is black hole if it has mass $3\mathcal{M}_{\odot}$\\
Gravitational lensing: Multiple images or rings of distant objects as light bends\\
Einstein's ring: Seen when massive object is spherical \quad Einstein's cross: Seen when massive object has less regular shape\\
Gravitational waves: Massive objects undergoing rapid non-uniform motion emit gravitational waves (Binary black hole system)\\
Gamma ray bursts origin: Supernovae, hypernova, collision between neutron stars
\section{Our Galaxy}
\textit{Galaxy}: Spiral galaxy with (on average $10^{11}$) stars in space, all held together by gravity and orbiting a common center.\\
Milky Way consists of disk with spiral arms, nuclear bulge and halos with globular clusters \quad $d_{\text{Milky Way}}=25000\pc=75000\ly$\\
Sun is located about $27000\ly$ from galactic center\\
Disk: Contains gas and dust, hosts younger generation of stars, location of open clusters \quad Bulge: Mixture of both young and old stars \\
Halo: Contains very few gas or dust, hosts older generation of stars, location of globular clusters, a lot of dark matter\\
Star in disk orbit same direction with up-and-down motion due to gravity of disk stars pulling them toward the disk\\
Orbits of stars in bulge and halo have random orientation\\
Orbital Velocity Law: $\mathcal{M}_{r}=\frac{r\times v^{2}}{G}=\frac{4\pi^{2}r^{3}}{Gp^{2}}$ \quad Mass within Sun's orbit: $1\times 10^{11}\mathcal{M}_{\odot}$\\
Galactic Recycling: Star-Gas-Star cycle \quad Gas from old stars form new stars\\
Multiple supernovae create giant bubble in which hot gas can blow out of it into halo \quad Gas clouds cooling rain back down onto disk\\
Atomic H gas forms as hot gas cools allowing electrons to join with protons\\
Molecular clouds form after gas cools enough to allow atoms to combine into molecules \quad Stars form in molecular clouds\\
Radiation from newly formed stars erode surface of clouds and glows \quad Densest gas resist the erosion and continue to form stars\\
Observation of star-gas-star cycle:\\
$21\cm$ radio waves show where H gas cooled and settled in disk \quad $2.6-1.3\mm$ radio waves from CO show locations of molecular clouds\\
Long $\lambda$ infrared ($60-100\mum$) from star heated intestellar dust \quad Short $\lambda$ infrared ($1-4\mum$) shows stars behind interstellar materials\\
Visible light emitted by stars is scattered and absorbed by dust \quad X-ray emitted from hot gas bubbles and X-ray binaries (Point-like)\\
Gamma-ray emitted from collisions of cosmic rays with atomic nuclei in interstellar clouds\\
$21\cm$ radio radiation: Electron orbiting proton with parallel spins has higher energy than one with anti-parallel\\
Hyperfine structure arises from coupling between magnetic moment of electron and nuclear magnetic moment\\
$2.6-1.3\mm$ radio waves: Comes from rotational transition ($3$ rotational energy levels: 2 emit $1.3\mm$, $2.6\mm$ to 0)\\
Easily detected CO radio emission lines are used to infer amount of H$_{2}$ (Only in Milky Way)\\
Emission nebulae, blue stars $\rightarrow$ O-going star formation \quad Star formation mostly happens in spiral arms\\
Spiral arms: Not fixed \quad Gas clouds get squeezed as they move into spiral arms $\rightarrow$ Star formation $\rightarrow$ Young stars\\
Center of Milky Way: Sagittarius A$^{*}$ \quad Using radio frequency, infra and x-ray imaging with adaptive optics to identify center with precision and clarity\\
Enclosed mass density is too high to be compatible with anything other than single supermassive black hole ($4.3\times 10^{6}\mathcal{M}_{\odot}$)
\section{Galaxies, the Foundation of Modern Cosmology and Galaxy Evolution}
Cosmology: Study of structure and evolution of universe\\
Spiral Galaxy: Disk component: Stars of all ages, many gas clouds\\
Bulge: Mixture of both young and old stars \quad Halo: Old stars, globular clusters, few gas clouds, lots of dark matter\\
Barred spiral galaxy: Bar of stars across the bulge\\
Elliptical galaxy: Spheroidal shape, no disk component\\
Lenticular galaxy: Has disk like spiral galaxy, but much less dusty gas and lack spiral arms (Intermediate between spiral and elliptical)\\
Irregular galaxy: No particular shape\\
Spiral galaxies found in groups of galaxies \qquad Elliptical galaxies found in huge clusters of galaxies\\
Cosmic Distance Ladder: Stellar parallax limit to $200\pc$, use properties of nearby stars for further stars\\
Radar ranging ($1\AU$), stellar parallax ($200\pc$), spectroscopic parallax ($10000\pc$), variable stars ($50\times 10^{6}\pc$)\\
Variable stars: Standard candle: Object with known luminosity\\
Variable stars pulsate because of changing balance between thermal pressure and gravity near equilibrium configuration\\
Happens in energy absorbing layer in outer envelope where He is partially ionized \\ 
Absorbing and releasing of energy lead to oscillation of star size, hence $\mathcal{L}$ and surface $\mathcal{T}$\\
Stars with higher $\mathcal{T}$: Layer cannot be formed \quad Stars with lower $\mathcal{T}$: Layer locate very deep in envelope, not significant variation\\
Period of variable star tells average $\mathcal{L}$, we can use the stars as standard candles\\
Apparent brightness of white-dwarf supernova tell us distance of its galaxy ($3066\times 10^{6}\pc$) \\
Type Ia Supernova always give same amount of energy ($M=-19.3$)\\
Hubble: Virtually all galaxies are redshifted (Moving away from us) \quad Redshift and distance are related\\
Hubble's Law: $d=\frac{v}{H}$ (Hubble constant $H=64000\m\s^{-1}\Mpc^{-1}$) \quad Implies expanding universe\\
Universe age $t=\frac{d}{v}=\frac{1}{H}$ \quad Cosmological Principle: Universe looks about the same on very large scales no matter where you are in it\\
Matter is almost evenly distributed on very large scale \quad No center and no edge\\
Cosmological redshift: Expansion of universe stretches photon $\lambda$\\
Galaxy formation: Start from protogalactic clouds, H and He gases formed first stars\\
Supernova heated surrounding interstellar gas, slowed down collapse of protogalactic clouds \\
For spiral galaxies, leftover gas settled into spinning disk\\
High $\mathcal{D}$ cloud form stars faster $\rightarrow$ Elliptical galaxy \quad Low $\mathcal{D}$ cloud form stars slower $\rightarrow$ Spiral galaxy\\
Galactic Cannibalism: Massive galaxies eat small ones due to larger gravity (Milky Way eats up large and small magellanic clouds)\\
Galactic collisions: More likely early in time \quad Trigger burst of star formation \quad Form elliptical galaxies\\
Starburst galaxies: From collision or near collision \quad Form stars so quickly that they use up all gas in less than billion yrs\\
Redshift $z=\frac{\lambda_{\text{shift}}-\lambda_{\text{rest}}}{\lambda_{\text{rest}}}$ \qquad Recessional speed $v_{r}=c\frac{(z+1)^{2}-1}{(z+1)^{2}+1}$\\
Quasars: $>10^{12}\mathcal{L}_{\odot}$ \quad Vary in brightness on timescale of hours, days or weeks\\
Highly redshifted spectra of quasars indicates large distance \quad Contains very wide range of $\lambda$ (Contain matter with wide range of $\mathcal{T}$)\\
Active galactic nucleus (AGN): Unusually bright center of galaxy\\
Normal galaxy: $>90\%$ of galaxies, $10^{6}-10^{10}\mathcal{L}_{\odot}$ \\ Active galaxy: $>10^{12}\mathcal{L}_{\odot}$, strong radiation from radio frequency to x-ray\\
Radio galaxy: Contain active nuclei shooting out jets of plasma that power lobes' radio radiation through synchrotron radiation\\
Charged particles in synchrotron radiation moves near speed of light but decelerating due to interaction with intergalactic medium\\
AGN shoot out blobs of plasma at speed of light $\rightarrow$ supermassive black hole is present\\
Accretion disks of some radio galaxies blocked by dusty gas clouds, look like quasar viewed along direction closer to jet axis\\
Power source of quasars and AGN: Accretion of gas onto a supermassive black hole at the center
\newpage
\section{Birth of Universe, Dark Matter, Dark Energy, and Fate of Universe}
Age of universe: about $13.8\times 10^{9}\yrs$, but due to expansion of space, humans are observing objects that were originally much closer but are now considerably farther away.\\
Current Cosmic Model: Dark Matter $22\%$, dark energy $73\%$\\
Normal matter: $10\g$ in $10^{30}\cm^{3}$ \quad Dark matter: Matter we infer to exist through gravitational effects but emit no detectable radiation\\
Evidence of existence:
$\mathcal{M}$ of spiral galaxies: Far more $\mathcal{M}$ than we see in stars \quad Flat rotational curves meaning lots of dark matter\\
$\mathcal{M}$ of elliptical galaxies: More massive galaxies have more gravity so stars are moving at faster orbital speed\\
More massive galaxies should have broader absorption lines\\
$\mathcal{M}$ in galaxy clusters: Measure velocities of galaxies from Doppler shift, $\mathcal{T}$ of hot gases, gravitational lensing\\
Possible dark matter: Massive Compact Halo Objects, weakly interacting massive particles, axions and neutrino-like particles\\
Olber's Paradox: If Universe if infinite, unchanging and everywhere the same, then night sky should be bright\\
Since night sky is dark, universe cannot be unchanging in time and infinite in space\\
Big bang: Hot and dense universe starts to expand\\
Earliest phases: Consists of radiation (photons) at $\mathcal{T}=10^{32}\K$ and high $\mathcal{D}$. \\
$1\s$: $\mathcal{T}=10^{10}$, protons, neutrons, electrons, neutrino emerged\\
Primordial Nucleosynthesis: H, He, Li, Be and isotopes formed\\
Cooler further: Formation of stars and galaxies (Formed He, C and heavier elements)\\
Hubble's Law: Redshift due to expansion of space-time\\
Cosmic Background Radiation (CBR): Freely streaming across universe since atoms formed at $\mathcal{T}=3000\K$ (Become transparent)\\
When early universe was hot, it was radiation-dominated (Mostly electrons, ions, scattered photons), makes it opaque\\
When it expands, at age of $380000\yrs$, cools enough for recombination of ions and electrons to form neutral atoms (Become transparent)\\
COBE: Satellite/thermometer to measure diffuse infrared and microwave radiation from early universe\\
BB model: Total expansion is $1000$ time since universe became transparent \quad Measured $\mathcal{T}=3\K$ \quad Observed $=2.735\K$\\
Model also predict spectrum obey black-body radiation form\\
Early universe: Photon converted to particle-antiparticle pairs \\
At $10^{9}\K$, protons and neutrons fused into deuterium and helium \quad $2$ p $+2$ n $\rightarrow 1$ $^{4}$He \quad converted all free neutrons into He in $5$ minutes\\
In very short time of He synthesis, there were $7$ protons every neutrons (Explain why H/He mass abundance ratio is $3$)\\
Model also works extremely well for other light elements \quad Form $^{2}$H, $^{4}$He, $^{3}$He, $^{6}$He, $^{7}$He in $5$ minutes\\
BB model: basic framework of modern cosmology\\
Curvature of Universe: Universe cannot be perfectly uniform (Inhomogeneities show in CBR)\\
Angular size of $\mathcal{T}$ variation used to determine curvature of Universe ($1\degree$ in angular size for flat universe) Data support Universe is flat\\
At age $10^{-32}\s$, Universe suddenly inflated by factor of $10^{78}$ due to high-energy to low-energy transition\\
Force of nature separated around cosmic inflation \quad Small size of Universe before inflation allow $\mathcal{T}$ to equalize $\rightarrow$ isotropic\\
Flatness Problem: Inflation theory means tremendous expansion greatly dilutes initial curvature\\
Curvature of Universe $k=-1,0,1$ depends on total mass density $\rho$ \quad Critical density for $k=0$ $\rho_{c}=\frac{3H_{0}^{2}}{8\pi G}=10^{-29}\g\cm^{-3}$\\
Density normalized to critical density $\ohm=\frac{\rho}{\rho_{c}}$\\
Dark energy: Causing universe expansion to accelerate (no idea what it is) \quad Anti-gravity\\
Models without acceleration: Assume $\ohm_{m}=1$ implies age only $9\times 10^{9}\yrs$ (Younger than some globular clusters)\\
Total energy density $\ohm_{t}=\ohm_{m}+\ohm_{k}+\ohm_{\Lambda}$ \quad $m$: Ordinary and extraordinary matter \quad $k$: curvature of space \quad $\Lambda$: Dark energy\\
Since curvature of universe is flat, $\ohm_{k}=0$ \quad Geometrically flat: $\ohm_{t}=1.02\pm 0.02$\\
Mass of universe: $\ohm_{m}=\ohm_{DM}+\ohm_{\text{atom}}\approx 0.3$ \quad This means $\ohm_{\Lambda}\approx 0.7$\\
Concordance Cosmic Model ($\Lambda$CDM Model):\\
Type Ia supernovae: Cosmic acceleration depends on difference between repulsive $\ohm_{\Lambda}$ and attractive $\ohm_{m}$, $\ohm_{\Lambda}>\ohm_{m}$\\
Galaxy clusters: $\ohm_{m}\approx 0.3$ \quad CBR Ripples: $\ohm_{m}+\ohm_{\Lambda}=1$\\
Before dark energy: $\ohm_{m}>1$: Closed universe $\rightarrow$ Big crunch \quad $\ohm_{m}<1$: Open universe $\rightarrow$ Big freeze\\
After dark energy: $\ohm_{m}\approx 0.3$, $\ohm_{\Lambda}\approx 0.7$: Flat and accelerating universe $\rightarrow$ Big Freeze
\end{document}
